\documentclass{article}
\usepackage{metric-min}
%\usepackage{showkeys}

\begin{document}

\title{Monotonicity of saddle maps}
\author{Anton Petrunin and Stephan Stadler}
%\address{A. Petrunin\newline\vskip-4mm Math. Dept. PSU,University Park, PA 16802,USA}
%\email{petrunin@math.psu.edu}
%\address{S. Stadler\newline\vskip-4mm Math. Inst.,Universit\"at M\"unchen, Theresienstr. 39, D-80333 M\"unchen, Germany}\email{stadler@math.lmu.de}
%\thanks{A.~Petrunin was partially supported by NSF grant DMS 1309340.}


\date{}

\maketitle

\begin{abstract}
We prove an analog of the univalentness theorem for saddle maps from disc to disc.
\end{abstract}

\section{Introduction}

A map from a closed disc $\DD$ to a Euclidean space is called saddle if for any hypeplane $\Pi$ each connected component of the complement $\DD\backslash f^{-1}\Pi$ intersects the boundary $\partial \DD$.
Ruled surfaces and the harmonic maps provide examples of such maps.

In this note we prove a synthetic analog of the univalentness theorem for saddle maps from disc to disc.
The original theorem is formulated for harmonic maps and  surfaces with nonpositive curvature.
It was proved by Richard Schoen and Shing Tung Yau \cite{schoen-yau};
an interesting generalization was obtained by J\"urgen Jost \cite{jost}.
An extensive study of general saddle maps was given by Samuil Shefel \cite{shefel-2D} and \cite{shefel-3D};
his work was inspired by a note of Alexander Alexandrov \cite{A} on intrinsic metric of general ruled surface.
Part of the work of Samuil Shefel is described in a chapter of \cite{akp}.

A continuous map $f\:X\to Y$ is called \emph{light} if the inverse image of any point $y\in Y$ is totally disconnected.

\begin{thm}{Baby theorem}\label{baby}
Let $f\:\DD\to \DD$ be a light saddle map.
Assume that 
the restriction $f|_{\partial\DD}$ is the identity map.
Then $f$ is a homeomorphism.
\end{thm}

The proof of the baby theorem is as hard as the proof of the main theorem formulated below --- read the final remarks if it looks trivial for you.

A map $f$ from a closed disc $\DD$ to a surface $\Delta$ with geodesic metric is called saddle if for any geodesic $[x,y]$ each connected component of the complement $\DD\backslash f^{-1}[x,y]$ meets the boundary $\partial\DD$.
It is easy to see that this definition agrees with the one given above.

A continuous  $f\:X\to Y$ is called \emph{monotonic} if the inverse image of any point $y\in Y$ is connected.
Note that according to the definition any monotonic map is onto.

\begin{thm}{Main theorem}\label{thm:main}
Let $\Delta=(\DD,|{*}-{*}|)$ be a closed disc equipped with a $\CAT[\kappa]$ metric such that any two points are joined by a unique geodesic and it depends continuously on the end points.
Assume $f\:\DD\to \Delta$ is a saddle map and the restriction of $f$ to the boundary $\partial\Delta$ is a monotonic map $\partial\DD\to  \partial\Delta$.
Then $f$ is monotonic. 

If in addition $f$ is light then it is a homeomorphism.
\end{thm}

In other words, the theorem states that monotonicity as an approprate generalization of univalentness which works for saddle maps.

Note that any disc with $\CAT[0]$ metric satisfies the assumption for $\Delta$.
In particular the it includes the nonpositively curved surfaces considered in the original univalentness theorem in \cite{schoen-yau}.
The surfaces described in \cite{jost} also partial case of these spaces.

\begin{thm}{Corollary}
by Let $\Sigma$ be a saddle surface in $\RR^3$ homeomorphic to a disc.
Assume that orthogonal projection to $(x,y)$-plane
maps the boundary of $\Sigma$
injectively to convex closed curve.
Then the orthogonal projection to $(x,y)$-plane is injective on whole $\Sigma$.

In particular, $\Sigma$ is a graph $z=f(x,y)$ for a function $f$ defined on a convex figure in the $(x,y)$-plane.
\end{thm}

To prove the corollary one has to project $\Sigma$ to $(x,y)$-plane and apply the main theorem.
This corollary is a generalization of the problem ``Saddle surface'' in \cite{petrunin-orthodox}.
Together with Shefel's theorem (see \cite{shefel-3D} and \cite[4.5.5]{akp}) it implies that the induced intrinsic metric on $\Sigma$ satisfies the $\CAT[0]$ comparison. 

\section{Energy minimizing maps are saddle}

Recall that harmonic map $f\:M\to N$ between Riemannian manifolds are defined as local energy minimizer among the maps with fixad values on the boundary.
Here the energy is defined as 
\[E(f)=\int_M|df|^2,\]
where $df\:\T M\to \T N$ is the differential of $f$.

\begin{thm}{Proposition} 
Assume $(\DD,|{*}-{*}|)$ is a disc with Riemannian metric such that any two points $x,y\in\DD$ are jointed by unique geodesic $[x,y]$.
Then any energy minimizing harmonic map $f\:\DD\to(\DD,|{*}-{*}|)$ is saddle.
\end{thm}

\parit{Proof.}
Assume contrary, that is for some geodesic $[x,y]$ in $(\DD,d)$ the complement $\DD\backslash f^{-1}[x,y]$ has a component $\Omega$ which does not meet the boundary $\partial\DD$.

Let $\gamma(t)$ be the unit speed parametrization of the geodesic $[x,y]$ from $x$ to $y$.
Let us redefine the map $f$ in $\Omega$ by setting 
\[\hat f(z)\left[
\begin{aligned}
&\gamma(\min\{\,|x-z|,|y-z|\,\})&&\text{if}&& z\in\Omega,
\\
&f(z)&&\text{if}&& z\notin\Omega.
\end{aligned}
\right.\]
Note that $E(f)>E(\hat f)$ and $f|_{\partial \DD}\equiv \hat f|_{\partial \DD}$ a contradiction.
\qeds


\section{Proofs}

\begin{thm}{Claim}\label{claim}
Let $f\:\DD\to \Delta$ be as in the main theorem.
Then for any closed convex set $K\subset\Delta$ each connected component of $\DD\backslash f^{-1}K$ intersects $\partial\DD$.
\end{thm}

\parit{Proof.}???\qeds

\parit{Proof of the main theorem.}
Since $f|_{\partial\DD}\:\partial\DD \to\partial \Delta$ is monotonic, it has degree $\pm1$.
We can assume that the orientations on $\DD$ and $\Delta$ are chosen so that $\deg f|_{\partial\DD}=1$
and therefore $\deg f=1$;
in particular $f$ is onto.

Assume $f$ is not monotone;
that is, there is a point $x\in \DD$ such that the inverse image $f^{-1}\{x\}$ is not connected.

Given $s\in\partial \Delta$ consider the open set
\[\Phi_s=\set{y\in\Delta}{\measuredangle [x\,^y_s]<\tfrac\pi2}.\]
Note that $\Phi_s$ and its complement are convex in $\Delta$.
In particular $\partial_\Delta\Phi_s$ is a geodesic.

Consider the two open subsets $\Theta\subset \DD$ and $\Omega\subset  \DD\times\SS^1$ defined as
\begin{align*}
\Theta&=\DD\backslash f^{-1}\{x\},
\\
\Omega&=\set{(z,s)\in \DD\times\SS^1}{f(z)\in \Phi_s}.
\end{align*}

The  projection $\DD\times\SS^1\to \DD$ sends $\Omega$ to $\Theta$.
Note that induced homomorphism $\pi_1\Omega\to \pi_1\Theta$ is onto.
Indeed, since $f|_{\partial\DD}$ is monotonic, any loop $\alpha$ in $\Theta$ can be reparametrized so that there is a curve $\beta\:[0,1]\to \partial\DD=\SS^1$ such that 
$f\circ\alpha(t)$ lies on the geodesic $[x,f\circ\beta(t)]$ in $\Delta$.
Then $\tilde\alpha(t)=(\alpha(t),\beta(t))$ is a lift of $\alpha$ to $\Omega\subset\DD\times\SS^1$.

Since $\{x\}$ is convex, by the claim \ref{claim}, every connected component of $\Theta$ have to intersect $\partial\DD$.
Note that $\Theta\cap \partial\DD$ is connected;
indeed if $x\notin \partial\DD$, then $\Theta\cap \partial\DD=\partial \DD$
and if $x\in \partial\DD$, then since $f|_{\partial\DD}$ is monotonic, $\Theta\cap \partial\DD$ is an open arc.
It follows that $\Theta$ is connected as well.


Consider the restriction of the projection $\DD\times\SS^1\to \SS^1$ to $\Omega$;
it has fiber $\Psi_s=f^{-1}\Phi_s$ at point $s\in\SS^1$.
Let us show that for any $s$ the set $\Psi_s$ is empty or simply connected.

Fix $s\in\SS^1$; assume $\Psi_s\ne \emptyset$.
Since $\Psi_s$ is a complement of a closed convex set,
each connected component must meet $\sigma_s=\Psi_s\cap\SS^1$.
Since $f$ is monotonic, $\sigma_s$ is an open arc in $\SS^1$.
In particular $\sigma_s$ is connected and therefore so is $\Psi_s$.

Choose a simple closed curve $\gamma\:\SS^1\to\Psi_s$;
denote by $\Gamma$ the disc bounded by $\gamma$.
By the claim \ref{claim}, $f(\Gamma)$ lies in the convex hull $K\subset \Delta$ of $f(\gamma)$.
Since $\Phi_s$ is convex,  $K\subset \Phi_s$.
It follows that $\Gamma\subset\Psi_s$ and therefore $\gamma$ is contractible in $\Psi_s$.
Since $\gamma$ is arbitrary, $\Psi_s$ is simply connected.

Note that $\Phi_s=\emptyset\iff f(s)=x$.
\begin{itemize}
\item If $x\notin\partial\DD$ then from above $\Phi_s$ is simply connected for any $s$.
Therefore the projection $\Omega\to \SS^1$ induce an isomorphism of fundamental groups; that is $\pi_1\Omega=\ZZ$.
\item If $x\in\partial\DD$ then by a similar reason, we have $\pi_1\Omega=0$.
\end{itemize}



Since $f^{-1}\{x\}$ is not connected, it can be divided into two subsets by a curve in $\Theta$.
\begin{itemize}
\item If $x\notin\partial \DD$, it follows that $\pi_1\Theta$ and therefore $\pi_1\Omega$ contain a free group with two generators, a contradicition.
\item If $x\in\partial\DD$, it follows that $\pi_1\Theta$ and therefore $\pi_1\Omega$ contain $\ZZ$ as a subgroup, a contradicition again.\qeds
\end{itemize}


\section{Final remarks}

\parbf{Proofs of baby theorem.}
Let us present a \emph{fake} of the baby theorem (\ref{baby}) which is based on the idea from the problem ``Saddle surface'' in \cite{petrunin-orthodox}.
We say where we cheat in the footnote; 
it is hard (if at all possible) to fix it.

\parit{Fake proof.}
Note that  $\deg f=1$;
in particular $f$ is onto.
It remains to show that $f$ is injective.

Assume  $w=f(x)=f(y)$ for distinct points $x,y\in\DD$
Note that  $w$ lies in the interior of $\DD$.
Choose a geodesic $\gamma$ which passes through $w$ and goes 
from boundary to boundary of $\DD$.
The inverse image $p^{-1}(\gamma)$ is a contractible set with two ends at $\partial\|\DD\|_s$, say $a$ and $b$.
We can assume that the points $a,x,y,b$ appear in the same order on $p^{-1}(\gamma)$.\footnote{This is where we are cheating: the inverse image $p^{-1}(\gamma)$ might be as terrible as psedoarc, where the order points has no sense.}

Note that there is a continuous one parameter family of geodesics $\gamma_t$ passing through $w$ with the ends at $\partial \DD$
such that $\gamma=\gamma_0$ and $\gamma_1$ is $\gamma$ with reversed parametrization.
Note that the order of $x$ and $y$ on $p^{-1}(\gamma_t)$ does not change in $t$.
On the other hand the orders on $\gamma_0$ and on $\gamma_1$ are opposite, a contradiction.\qeds

A correct proof of baby theorem can be build on the theorem of Shefel, but it does not seem possible to generalize.

\parit{Proof.}
Let us extend the map of the disc by identity map outside the disc. 
According to Shefel's theorem the induced metric length metric on the plane is $\CAT[0]$.
This metric coinside with the Euclidean metric outside a compact set therefore the induced intrinsic metric metric is flat and the maps is isometry for this metric, in particular a homeomorphism.\qeds


\parbf{Generalization of main theorem.}
In the proof the condition that $\DD$ is a disc can be relaxed to the following:
the mapping cylinder of the target space over $f|_{\partial\DD}$ is homeomorphic to a closed disc.
So the target space might look like the solid figure eight on the right picture.

\begin{center}
\begin{lpic}[t(-2 mm),b(-0 mm),r(0 mm),l(0 mm)]{pics/mapping-cylinder(1)}
\lbl[rb]{41,17;$\xrightarrow{\hat f}$}
\end{lpic}
\end{center}

\parbf{On univalentness of harmonic maps.}
If $f\:\DD\to \Sigma$ is a harmonic map from a closed disc to surface with Riemannian metric,
then one can show that for any $y\notin f(\partial\DD)$ the inverse image $f^{-1}\{y\}$ is a descrete set of points.
A proof of this statement was suggested by Alexandre Eremenko.

We do not know if this statement can be generalized to surfaces with $\CAT[0]$ metric; 
for a definition of harmonic maps with $\CAT[0]$ target see \cite{GS}.
If yes, then it would imply that if the  map $f$ is harmonic and univalent at the image of $f(\partial\DD)$ then it is injective.

\begin{thebibliography}{52}

\bibitem{A} Alexandrov, A. D. ``Ruled  surfaces  in  metric  spaces,'' Vestnik Leningrad. Univ., 12:5-26, 1957 (Russian).

\bibitem{eremenko} Eremenko, A.,  
an answer to ``Harmonic maps are light'', MathOverflow
\texttt{https://mathoverflow.net/q/272047 (version: 2017-06-13)}

\bibitem{akp}
Alexander, S., Kapovitch, V. and Petrunin, A.,
``Invitation to Alexandrov geometry: CAT [0] spaces,''
arXiv:1701.03483.

\bibitem{GS} Gromov, Mikhail, and Richard Schoen. "Harmonic maps into singular spaces and p-adic superrigidity for lattices in groups of rank one." Publications Mathématiques de l'IHÉS 76.1 (1992): 165-246.

\bibitem{H} Hamilton, R. S. ``Harmonic Maps of Manifolds with Boundary,'' Lecture Notes in Mathematics, Springer, 1975, ISBN 978-3-540-37530-2.

\bibitem{jost} Jost, J.
Univalency of harmonic mappings between surfaces.
J. Reine Angew. Math. 324 (1981), 141--153. 

\bibitem{moore}
Moore, R. L.,
``Concerning upper semi-continuous collections of continua,''
Trans. Amer. Math. Soc. 27 no. 4 (1925) pp. 416--428.

\bibitem{petrunin-orthodox} Petrunin, A. 
``Exercises in Orthodox Geometry''
{\tt arXiv:0906.0290 [math.HO]}

\bibitem{shefel-2D} 
\begin{otherlanguage}{russian}
Шефель, С. З.,
\textit{О седловых поверхностях ограниченной спрямляемой кривой.}
Доклады АН СССР, 162 (1965) №2, 
294---296.
\end{otherlanguage}
%\v{S}efel', S.,
%\textit{On saddle surfaces bounded by a rectifiable curve,} 
%Dokl. Akad. Nauk SSSR 
%162 
%(1965), 
%294--296.

\bibitem{shefel-3D} 
\begin{otherlanguage}{russian}
Шефель, С. З., 
\textit{О внутренней геометрии седловых поверхностей.}
Сибирский математический журнал, 5 (1964), 1382---1396
\end{otherlanguage}
%\v{S}efel', S., 
%\textit{On the intrinsic geometry of saddle surfaces,} Sibirsk. Mat. \v{Z}. 
%5 
%(1964), 
%1382--1396

\bibitem{schoen-yau} Schoen, Richard; Yau, Shing Tung
On univalent harmonic maps between surfaces.
Invent. Math. 44 (1978), no. 3, 265--278. 

\bibitem{St} Stadler, S. ``Harmonic discs in CAT(0) spaces'', in preparation.

%\bibitem{W1}Whyburn, G. T., ``On sequences and limiting sets,'' Fund. Math. vol. 25 (1935) pp. 408-426.

%\bibitem{W2}Whyburn, G. T., ``Analytic topology,'' Amer. Math. Soc. Colloquium Publications, vol. 28, 1942.

%\bibitem{Wi}Wilder, R. L., ``Topology of Manifolds,'' American Mathematical Society Colloquium Publications, vol. 32. American Mathematical
%Society, New York, N. Y., 1949.
\end{thebibliography}

\end{document}