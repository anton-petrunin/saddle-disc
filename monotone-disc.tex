\documentclass{article}
\usepackage{metric-min}
%\usepackage{showkeys}

\begin{document}

\title{Monotonicity of saddle maps}
\author{Anton Petrunin and Stephan Stadler}
%\address{A. Petrunin\newline\vskip-4mm Math. Dept. PSU,University Park, PA 16802,USA}
%\email{petrunin@math.psu.edu}
%\address{S. Stadler\newline\vskip-4mm Math. Inst.,Universit\"at M\"unchen, Theresienstr. 39, D-80333 M\"unchen, Germany}\email{stadler@math.lmu.de}
%\thanks{A.~Petrunin was partially supported by NSF grant DMS 1309340.}


\date{}

\maketitle

\begin{abstract}
We prove an analog of Schoen--Yau univalentness theorem for saddle maps between discs.
\end{abstract}

\section{Introduction}

A map from a closed disc $\DD$ to a Euclidean space is called \emph{saddle} if for any hyperplane $\Pi$ each connected component of the 
complement $\DD\backslash f^{-1}\Pi$ intersects the boundary $\partial \DD$.
Ruled surfaces and harmonic maps provide examples of such maps.

In this note we prove a synthetic analog of the univalentness theorem for saddle maps from disc to disc.
The original theorem is formulated for harmonic maps and  surfaces with nonpositive curvature.
It was proved by Richard Schoen and Shing Tung Yau \cite{schoen-yau};
an interesting generalization was obtained by J\"urgen Jost \cite{jost}.
An extensive study of general saddle maps was given by Samuil Shefel \cite{shefel-2D} and \cite{shefel-3D};
his work was inspired by a note of Alexander Alexandrov \cite{A} on the intrinsic metric of a general ruled surface.
Part of the work of Samuil Shefel is described in a chapter of \cite{akp}.

A continuous map $f\:X\to Y$ is called \emph{light} if the inverse image of any point $y\in Y$ is totally disconnected.

\begin{thm}{Baby theorem}\label{baby}
Let $f\:\DD\to \DD$ be a light saddle map.
Assume that 
the restriction $f|_{\partial\DD}$ is the identity map.
Then $f$ is a homeomorphism.
\end{thm}

A map $f$ from a closed disc $\DD$ to a surface $\Delta$ with a geodesic metric is called \emph{saddle} 
if for any geodesic $[x,y]$ in $\Delta$, each connected component of the complement $\DD\backslash f^{-1}[x,y]$ meets the boundary $\partial\DD$.
It is easy to see that this definition agrees with the one given above.

A continuous map $f\:X\to Y$ is called \emph{monotonic} if the inverse image of any point $y\in Y$ is connected.
Since connected space is nonempty by definition, any monotonic map is onto.

\begin{thm}{Main theorem}\label{thm:main}
Let $\Delta=(\DD,|{*}-{*}|)$ be a closed disc equipped with a $\CAT[\kappa]$ metric 
such that any two points are joined by a unique geodesic and it depends continuously on the end points.
Assume $f\:\DD\to \Delta$ is a saddle map and the restriction of $f$ to the boundary $\partial\Delta$ is a monotonic map 
$\partial\DD\to  \partial\Delta$.
Then $f$ is monotonic. 
\end{thm}

In other words, the theorem states that monotonicity is an appropriate generalization of univalentness which works for saddle maps.

\parbf{Remarks.}

\begin{itemize}
\item Note that any disc with $\CAT[0]$ metric satisfies the assumption for $\Delta$.
In particular, it includes the nonpositively curved surfaces considered in the original univalentness theorem in \cite{schoen-yau}.
The surfaces described in \cite{jost} are also partial case of these spaces.
\item For the saddle map $f\:\DD\to \Delta$ in the theorem,
one can use the so called \emph{monotone-light factorization} %need a ref to monotone-light factorization
which is $f=g\circ h$,
where $g$ is monotone and $h$ is light.
Moore's quotient theorem \cite{moore} implies that the target of the map $g$ is homeomorphic to the disc $\DD$, so we may think that $g\:\DD\to \DD$ and $h\:\DD\to \Delta$.
The light map $h$ is saddle and according to the theorem it is monotone.
The latter implies that $h$ is a homeomorphism.
\end{itemize}



\begin{thm}{Corollary}
Let $\Sigma$ be a saddle surface in $\RR^3$ homeomorphic to a disc.
Assume that the orthogonal projection to the $(x,y)$-plane
maps the boundary of $\Sigma$
injectively to a convex closed curve.
Then the orthogonal projection to the $(x,y)$-plane is injective on whole of $\Sigma$.

In particular, $\Sigma$ is a graph $z=f(x,y)$ for a function $f$ defined on a convex figure in the $(x,y)$-plane.
\end{thm}

To prove the corollary one has to project $\Sigma$ to the $(x,y)$-plane and apply the main theorem.

This corollary is a generalization of the problem ``Saddle surface'' in \cite{petrunin-orthodox}.
Together with Shefel's theorem (see \cite{shefel-3D} and \cite[4.5.5]{akp}) it implies that the induced 
intrinsic metric on $\Sigma$ satisfies the $\CAT[0]$ comparison. 
(It is not known whether any saddle surface in the Euclidean space has locally $\CAT[0]$ induced intrinsic metric.)

\section{Energy minimizing maps are saddle}

Recall that harmonic maps $f\:M\to N$ between Riemannian manifolds are defined as local energy minimizers 
among maps with fixed values on the boundary.
Here the energy is defined as 
\[E(f)=\int_M|df|^2,\]
where $df\:\T M\to \T N$ is the differential of $f$.

\begin{thm}{Proposition} 
Assume $\Delta=(\DD,|{*}-{*}|)$ is a disc with Riemannian metric such that any two points $x,y\in\Delta$ are joined by unique geodesic $[x,y]$.
Then any energy minimizing harmonic map $f\:\DD\to\Delta$ with fixed values on the boundary is saddle.
\end{thm}

\parit{Proof.}
Assume contrary, that is for some geodesic $[x,y]$ in $\Delta$ the complement $\DD\backslash f^{-1}[x,y]$ has a component 
$\Omega$ which does not meet the boundary $\partial\DD$.

Let $\gamma(t)$ be the unit speed parametrization of the geodesic $[x,y]$ from $x$ to $y$.
Let us redefine the map $f$ in $\Omega$ by setting 
\[\hat f(z)=
\left[
\begin{aligned}
&\gamma(\min\{\,|x-z|,|y-z|\,\})&&\text{if}&& z\in\Omega,
\\
&f(z)&&\text{if}&& z\notin\Omega.
\end{aligned}
\right.\]
Note that $E(f)>E(\hat f)$ and $f|_{\partial \DD}\equiv \hat f|_{\partial \DD}$, a contradiction.
\qeds


\section{The proof}

\begin{thm}{Claim}\label{claim}
Let $f\:\DD\to \Delta$ be as in the main theorem.
Then 
\begin{enumerate}[(i)]
\item for any closed convex set $K\subset\Delta$ each connected component of $\DD\backslash f^{-1}K$ intersects $\partial\DD$.
\item for any open convex set $\Phi\subset\Delta$ each connected component of $f^{-1}\Phi$ is simply connected.
\end{enumerate}
\end{thm}

\parit{Proof; (i)} 
Let $\gamma$ be a geodesic in $\Delta$, disjoint from $K$ and with endpoints on $\partial\Delta$.
Let $\Sigma_K$ be the set of all such geodesics.
 For each geodesic in $\Sigma_K$ we define
$H_\gamma$ to be the component of $\Delta\backslash\gamma$ which contains $K$. Then
\[K=\bigcap_{\gamma\in\Sigma_K} H_\gamma\]
since $K$ is closed and convex and $\Delta$ is homeomorphic to a disc.
In other words, if $x\notin K$ then there is a geodesic $\gamma$ which separates $K$ from $x$. %MORE???

By definition of saddle maps, each connected component of $\DD\backslash f^{-1}H_\gamma$ meets the boundary $\partial \DD$.
Therefore the same holds for the union
\[\DD\backslash f^{-1}K=\bigcup_{\gamma\in\Sigma_K}(\DD\backslash f^{-1}H_\gamma).\]

\parit{(ii)}
Choose a simple closed curve $\gamma\:\SS^1\to\Psi=f^{-1}\Phi$;
denote by $\Gamma$ the disc bounded by $\gamma$.
By \textit{(i)}, $f(\Gamma)$ lies in the convex hull $K\subset \Delta$ of $f(\gamma)$.
Since $\Phi$ is convex,  $K\subset \Phi$.
It follows that $\Gamma\subset\Psi$ and therefore $\gamma$ is contractible in $\Psi$.

Since $\gamma$ is arbitrary, $\Psi$ is simply connected.
\qeds

\parit{Proof of the main theorem.}
Since $f|_{\partial\DD}\:\partial\DD \to\partial \Delta$ is monotonic, it has degree $\pm1$.
We can assume that the orientations on $\DD$ and $\Delta$ are chosen so that $\deg f|_{\partial\DD}=1$
and therefore $\deg f=1$;
in particular $f$ is onto.

Assume $f$ is not monotone;
that is, there is a point $x\in \DD$ such that the inverse image $f^{-1}\{x\}$ is not connected.

Given $s\in\partial \Delta$ consider the open set
\[\Phi_s=\set{y\in\Delta}{\measuredangle [x\,^y_s]<\tfrac\pi2}.\]
Note that $\Phi_s$ and its complement are convex in $\Delta$.
(Here we use that $\Delta$ is homeomorphic to a disc; analogous statement does not hold higher dimensions.)
In particular, the relative boundary $\partial_\Delta\Phi_s$ is a geodesic.

Consider the two open subsets $\Theta\subset \DD$ and $\Omega\subset  \DD\times\SS^1$ defined as
\begin{align*}
\Theta&=\DD\backslash f^{-1}\{x\},
\\
\Omega&=\set{(z,s)\in \DD\times\SS^1}{f(z)\in \Phi_s}.
\end{align*}

The  projection $\DD\times\SS^1\to \DD$ sends $\Omega$ to $\Theta$.
Note that the induced homomorphism $\pi_1\Omega\to \pi_1\Theta$ is onto.
Indeed, for any point $z\in \Theta$ there is $s\in \SS^1$ such that $f(z)\in \Phi_s$ or equivalently $\measuredangle [x\,^{f(z)}_s]<\tfrac\pi2$.
Moreover, since $f|_{\partial\DD}$ is monotonic, the map the set of points $s$ satisfying the above condition is an open arc in $\SS^1$.
It follows that one can fix a continuous map $z\mapsto s_z$ such that $f(z)\in \Phi_{s_z}$ for any $z\in \Theta$.
It remains to note that for any loop $\alpha$ in $\Theta$, the loop $\tilde\alpha(t)=(\alpha(t),s_{\alpha(t)})$ is an $f$-lift of $\alpha$ to $\Omega\subset\DD\times\SS^1$.

Note that $\Theta\cap \partial\DD$ is connected.
Indeed,
\begin{itemize}
\item If $x\notin \partial\DD$, then $\Theta\cap \partial\DD=\partial \DD$. 
\item If $x\in \partial\DD$, then since $f|_{\partial\DD}$ is monotonic, $\Theta\cap \partial\DD$ is an open arc.
\end{itemize}
Since $\{x\}$ is convex, by Claim \ref{claim}\textit{(i)}, every connected component of $\Theta$ has to intersect $\partial\DD$.
It follows that $\Theta$ is connected as well.

Consider the restriction of the projection $\DD\times\SS^1\to \SS^1$ to $\Omega$;
it has fiber $\Psi_s=f^{-1}\Phi_s$ at the point $s\in\SS^1$.
By Claim \ref{claim} the set $\Psi_s$ is either empty or simply connected for any $s$.

Indeed, fix $s\in\SS^1$ and assume $\Psi_s\ne \emptyset$.
Since $\Psi_s$ is a complement of a closed convex set,
each connected component must meet $\sigma_s=\Psi_s\cap\SS^1$.
Since $f$ is monotonic, $\sigma_s$ is an open arc in $\SS^1$.
In particular $\sigma_s$ is connected and therefore so is $\Psi_s$.
Since $\Phi_s$ is convex open set, by Claim \ref{claim}\textit{(ii)}, $\Psi_s$ is simply connected.

Note that $\Phi_s=\emptyset\iff f(s)=x$.
\begin{itemize}
\item If $x\notin\partial\DD$ then from above $\Psi_s$ is simply connected for any $s$.
Therefore the projection $\Omega\to \SS^1$ induces an isomorphism of fundamental groups; that is $\pi_1\Omega=\ZZ$.
\item If $x\in\partial\DD$ then by a similar reason, we have $\pi_1\Omega=0$.
\end{itemize}



Since $f^{-1}\{x\}$ is not connected, it can be divided into two subsets by a curve in $\Theta$.
\begin{itemize}
\item If $x\notin\partial \DD$, it follows that $\pi_1\Theta$ and therefore $\pi_1\Omega$ contain a free group with two generators, a contradicition.
\item If $x\in\partial\DD$, it follows that $\pi_1\Theta$ and therefore $\pi_1\Omega$ contain $\ZZ$ as a subgroup, a contradicition again.\qeds
\end{itemize}


\section{Final remarks}

\parbf{Proofs of baby theorem.}
The presented proof is a tricky fix of the following \emph{fake} proof of the baby theorem (\ref{baby}).
We say where we cheat in the footnote; 
it is hard (if at all possible) to fix it.

\parit{Fake proof.}
Note that  $\deg f=1$;
in particular $f$ is onto.
It remains to show that $f$ is injective.

Assume  $w=f(x)=f(y)$ for distinct points $x,y\in\DD$.
Note that  $w$ lies in the interior of $\DD$.
Choose a chord $\gamma$ which passes thru $w$ and goes 
from boundary to boundary of $\DD$.
The inverse image $p^{-1}(\gamma)$ is a contractible set with two ends at $\partial\|\DD\|_s$, say $a$ and $b$.
We can assume that the points $a,x,y,b$ appear in the same order on $p^{-1}(\gamma)$.
\footnote{This is where we are cheating: the inverse image $p^{-1}(\gamma)$ might be as terrible as a pseudoarc, 
where the order of points has no sense.}

There is a continuous one parameter family of geodesics $\gamma_t$ passing thru $w$ with the ends at $\partial \DD$
such that $\gamma=\gamma_0$ and $\gamma_1$ is $\gamma$ with reversed parametrization.
Note that the order of $x$ and $y$ on $p^{-1}(\gamma_t)$ does not change in~$t$.
On the other hand the orders on $\gamma_0$ and $\gamma_1$ are opposite, a contradiction.\qeds

A correct proof of the baby theorem can be build on the deep theorem of Shefel \cite{shefel-2D}, but does not seem to be generalizable.

\parit{Proof.}
Let us extend the map of the disc by the identity map outside the disc. 
According to Shefel's theorem the induced length metric on the plane is $\CAT[0]$.
This metric coincides with the Euclidean metric outside a compact set 
therefore the induced intrinsic metric metric is flat and the map is isometric for this metric, in particular a homeomorphism.\qeds

\begin{wrapfigure}{r}{34 mm}
\begin{lpic}[t(-2 mm),b(-0 mm),r(0 mm),l(0 mm)]{pics/mapping-cylinder(1)}
%\lbl[rb]{41,17;$\xrightarrow{\hat f}$}
\end{lpic}
\end{wrapfigure}

\parbf{Generalization of main theorem.}
In the proof the condition that $\DD$ is a disc can be relaxed to the following:
the mapping cylinder of the target space over $f|_{\partial\DD}$ is homeomorphic to a closed disc.
So the target space $\Delta$ might look like the solid figure eight on the picture.


\parbf{On univalentness of harmonic maps.}
If $f\:\DD\to \Sigma$ is a harmonic map from a closed disc to a surface with a Riemannian metric,
then one can show that for any $y\notin f(\partial\DD)$ the inverse image $f^{-1}\{y\}$ is a discrete set of points.
A proof of this statement was suggested by Alexandre Eremenko, see \cite{eremenko}.

The corresponding statement for surfaces with $\CAT[0]$ metrics is wrong. 
Surprisingly there even exists a noninjective harmonic map from a closed disc to a $\CAT[0]$ disc
which restricts to a homeomorphism of boundaries, cf. \cite{Ku} and the example below. Therefore,
our main theorem is optimal even for harmonic maps.

\parbf{Example.}
Let $\Sigma$ be a closed surface of genus $g$. Choose a $\CAT[0]$ metric $ds^2$ on $\Sigma$
which is flat away from a finite number of cone points $C:=\{x_i\}_{i=1}^k$ where $k>4g-4$. 
Let $u:\Sigma\to (\Sigma,ds^2)$ be the unique harmonic map homotopic to the identity.
According to Theorem ? in \cite{Ku}, $u$ restricts to a diffeomorphism 
$\Sigma\backslash u^{-1}C\to \Sigma\backslash u^{-1}C$. Let $\Phi$ be the associated Hopf differential.
Since $k>4g-4$ there exists a cone point $x\in C$ such that $\Phi\neq 0$ on $u^{-1}(x)$. 
By Theorem ? in \cite{Ku}, $u^{-1}(x)$ is a simply connected graph in $\Sigma$. Then for $\epsilon>0$
small enough, $u^{-1} \overline{B_\epsilon(x)}$ is homeomorphic to a closed disc $D$ and $u$ restricts to
a homeomorphism on $\partial D$. If we precompose $u|_D$ with a Riemann mapping $\rho:\DD\to D$, then
we obtain a noninjective harmonic map between discs whose restriction to the boundary is a homeomorphism.

%If $\phi$ is a holomorphic differential, then each zero of order $m$ contributes a cone point of angle $(m+2)\pi$
%Hence by Gauss-Bonnet, the number of zeros is bounded by $4g-4$. 


\begin{thebibliography}{52}

\bibitem{A} Alexandrov, A. D. ``Ruled  surfaces  in  metric  spaces,'' Vestnik Leningrad. Univ., 12:5-26, 1957 (Russian).

\bibitem{eremenko} Eremenko, A.,  
an answer to ``Harmonic maps are light'', MathOverflow
\texttt{https://mathoverflow.net/q/272047 (version: 2017-06-13)}

\bibitem{akp}
Alexander, S., Kapovitch, V. and Petrunin, A.,
``Invitation to Alexandrov geometry: CAT [0] spaces,''
arXiv:1701.03483.

\bibitem{GS} Gromov, Mikhail, and Richard Schoen. "Harmonic maps into singular spaces and p-adic superrigidity for lattices in groups of rank one." Publications Mathématiques de l'IHÉS 76.1 (1992): 165-246.

\bibitem{H} Hamilton, R. S. ``Harmonic Maps of Manifolds with Boundary,'' Lecture Notes in Mathematics, Springer, 1975, ISBN 978-3-540-37530-2.

\bibitem{jost} Jost, J.
Univalency of harmonic mappings between surfaces.
J. Reine Angew. Math. 324 (1981), 141--153.

\bibitem{moore}
Moore, R. L.,
``Concerning upper semi-continuous collections of continua,''
Trans. Amer. Math. Soc. 27 no. 4 (1925) pp. 416--428.

\bibitem{Ku} Kuwert, E.
Harmonic maps between flat surfaces with conical singularities. 
Mathematische Zeitschrift 221 (1996), 421--436.



\bibitem{petrunin-orthodox} Petrunin, A. 
``Exercises in Orthodox Geometry''
{\tt arXiv:0906.0290 [math.HO]}

\bibitem{shefel-2D} 
\begin{otherlanguage}{russian}
Шефель, С. З.,
\textit{О седловых поверхностях ограниченной спрямляемой кривой.}
Доклады АН СССР, 162 (1965) №2, 
294---296.
\end{otherlanguage}
%\v{S}efel', S.,
%\textit{On saddle surfaces bounded by a rectifiable curve,} 
%Dokl. Akad. Nauk SSSR 
%162 
%(1965), 
%294--296.

\bibitem{shefel-3D} 
\begin{otherlanguage}{russian}
Шефель, С. З., 
\textit{О внутренней геометрии седловых поверхностей.}
Сибирский математический журнал, 5 (1964), 1382---1396
\end{otherlanguage}
%\v{S}efel', S., 
%\textit{On the intrinsic geometry of saddle surfaces,} Sibirsk. Mat. \v{Z}. 
%5 
%(1964), 
%1382--1396

\bibitem{schoen-yau} Schoen, Richard; Yau, Shing Tung
On univalent harmonic maps between surfaces.
Invent. Math. 44 (1978), no. 3, 265--278. 

\end{thebibliography}

\end{document}